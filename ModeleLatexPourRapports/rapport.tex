\documentclass[11pt,oneside,noprintercorrection]{ustl}

%----------------------------------------------------------------------
%                     Chargement de quelques packages
%----------------------------------------------------------------------

% Si l'on veut produire une version PDF avec distiller ou pdflatex :
%\usepackage{tlhypref}
\usepackage[pdfborder=0 0 0]{ustl}
% Si l'on produit le PDF avec pdflatex, ceci remplace la plupart
% des polices EC par des polices CM, plus adaptees a la generation de PDF,
% car ayant des equivalents PS :
% obsol�te ?? \usepackage{aeguill}


% Pour les figures PS :
\usepackage{graphicx}

% Si on veut des mini-tables des matieres (utiliser minitoc-hyper
% en conjonction avec tlhypref) :
\usepackage[french]{minitoc}
%\usepackage[french]{minitoc-hyper}

\usepackage[frenchb]{babel}
\usepackage{graphicx}
\usepackage{float}

% Pour les codes
\usepackage{listings}
\lstset{language=C++,basicstyle=\small}

\synctex = 1

%-------------------------------------------------------------------
%  Surcharge de commandes pour les variables et page d'en-t�te
%-------------------------------------------------------------------

\makeatletter

%
% les deux commandes suivantes sont entre \makeatletter
% et \makeatother parce qu'elles utilisent des `@'.
%

\renewcommand{\@DFD}{Universit� Lille 1}


\renewcommand{\@NancyIhe@d}{{\UseEntryFont{ThesisFirstPageHead}\noindent
    \centerline{\if@logo@uhp@
                    {\setbox0=\hbox{$\raise2.3cm\hbox{\UHPLogo}$}%
                     \ht0=\baselineskip\box0}\hfill
                \else
                    Universit� des Sciences et Technologies de Lille%
                \fi}%
    \@TL@cmn@head\\
    \par
    }%
    }


\newcommand\TheseLilleI{\renewcommand{\@ThesisFirstPageHead}{\@NancyIhe@d}%
                         \ThesisDiploma{{\UseEntryFont{ThesisDiploma}%
                              \\[3mm]
            {\UseEntryFont{ThesisSpecialty}( )}}}}

\makeatother

%-------------------------------------------------------------------
%           Corrections pour les imprimantes recto-verso
%                          (A AJUSTER)
%-------------------------------------------------------------------

%\ShiftOddPagesRight{-1mm}
%\ShiftOddPagesDown{2.5mm}
%\ShiftEvenPagesRight{0mm}
%\ShiftEvenPagesDown{0mm} 

%-------------------------------------------------------------------
%                Mise en page
%-------------------------------------------------------------------

%-------------------------------------------------------------------
%                             interligne
%-------------------------------------------------------------------
\renewcommand{\baselinestretch}{1.3}

%-------------------------------------------------------------------
%                             Marges
%-------------------------------------------------------------------

% pour positionner les vraies marges:
%\SetRealMargins{1mm}{1mm}

%-------------------------------------------------------------------
%                             En-tetes
%-------------------------------------------------------------------
%On n'utilise pas les logos
%\DontShowLogos

% Les en-tetes: quelques exemples
%\UppercaseHeadings
%\UnderlineHeadings
%\newcommand\bfheadings[1]{{\bf #1}}
%\FormatHeadingsWith{\bfheadings}
%\FormatHeadingsWith{\uppercase}
%\FormatHeadingsWith{\underline}
\newcommand\upun[1]{\uppercase{\underline{\underline{#1}}}}
\FormatHeadingsWith\upun

\newcommand\itheadings[1]{\textit{#1}}
\FormatHeadingsWith{\itheadings}

% pour avoir un trait sous l'en-tete:
\setlength{\HeadRuleWidth}{0.4pt}

%-------------------------------------------------------------------
%                Chemin d'inclusion des graphiques
%-------------------------------------------------------------------

\graphicspath{{img/}{./}}

%-------------------------------------------------------------------
%                         Les references
%-------------------------------------------------------------------

\NoChapterNumberInRef \NoChapterPrefix

%-------------------------------------------------------------------
%                           Brouillons
%-------------------------------------------------------------------

% ceci ajoute une marque `brouillon' et la date
%\ThesisDraft



\begin{document}
\renewcommand{\labelitemi}{$\bullet$}
\renewcommand{\labelitemii}{$\circ$}
%-------------------------------------------------------------------
%                          Encadrements
%-------------------------------------------------------------------

% encadre les chapitres dans la table des matieres:
% (ces commandes doivent figurer apres \begin{document}

%\FrameChaptersInToc
%\FramePartsInToc


%-------------------------------------------------------------------
%            Reinitialisation de la numerotation des chapitres
%-------------------------------------------------------------------

% Si la commande suivante est presente,
% elle doit figurer APRES \begin{document}
% et avant la premiere commande \part
\ResetChaptersAtParts

%-------------------------------------------------------------------
%               mini-tables des matieres par chapitre
%-------------------------------------------------------------------

% preparer les mini-tables des matieres par chapitre.
% (commande de minitoc.sty)
%\dominitoc

%-------------------------------------------------------------------
%                         Page de titre:
%-------------------------------------------------------------------


\ThesisKind{Rapport de projet XXX}
\ThesisPresentedThe{soutenu le XXXX}
\ThesisTitle{Rapport de stage}
\ThesisAuthor{Tony GALLOY}
%\NomDuLaboOuEntreprise{INRIA Equipe DEFROST}
%\LogoLaboOuEntreprise{alcove} % Image du logo du labo ou ets dans le rep img

\NomDuLaboOuEntreprise{}
\LogoLaboOuEntreprise{} % Image du logo du labo ou ets dans le rep img

\TheseLilleI

\NewJuryCategory{encadrant}{\it Encadrant :}
                        {\it Encadrants :}

%\encadrant = {Nom encadrant 1\\
%              Nom encadrant 2}

% Creation de la page de titre:
\MakeThesisTitlePage



%-------------------------------------------------------------------


%-------------------------------------------------------------------
%                          remerciements
%-------------------------------------------------------------------

%\DontFrameThisInToc
\begin{ThesisAcknowledgments}




\end{ThesisAcknowledgments}


%-------------------------------------------------------------------
%                  ecriture de `Chapitre' et `Partie'
%                      dans la table des matieres
%-------------------------------------------------------------------

\WritePartLabelInToc \WriteChapterLabelInToc

%-------------------------------------------------------------------
%                        table des matieres
%-------------------------------------------------------------------

\tableofcontents

%-------------------------------------------------------------------
%              Exemple d'utilisation de \SpecialSection
%-------------------------------------------------------------------

% La commande \mainmatter (nouvelle commande LaTeX2e) permet de passer
% a la numerotation arabe (ce que fait \pagenumbering{arabic})
% et de faire commencer la nouvelle page 1 sur une page impaire.
% On evitera donc d'utiliser directement \pagenumbering{arabic}.
\mainmatter

% ----------------------------------------------------------------
\SpecialSection{Introduction}



% Pour ne pas avoir le mot `Chapitre' au debut de chaque chapitre.
\NoChapterHead

XXXX


%--------------------------------------------------------------------------------------
%--------------------------------------------------------------------------------------
\chapter{Prise en main de \LaTeX}


\section{Installation}

\subsection{Sous Linux}
Nombreuses documentations disponibles sur internet pour
l'installation des packages (package texlive; pour l'�dition geany suffit).

\subsection{Sous Windows}

\begin{itemize}
\item Pour compiler les fichiers tex, installer Miktex \cite{Mikex}
\item Pour �crire des documents Latex, installer TeXnicenter \cite{Texnicenter} (Freeware) ou WinEdt \cite{WinEdt} (Shareware).
\end{itemize}

\section{Compilation des documents}
Pour compiler un document \LaTeX en pdf, il y a deux m�thodes possibles :

\begin{itemize}
\item La m�thode "traditionnelle": .tex -> .dvi avec latex, puis .dvi -> .ps (dvitops) puis .ps ->.pdf (pstopdf). Il est
 �galement possible de faire .tex -> .dvi avec latex, puis .dvi -> .pdf
(dvi2pdf). Cette m�thode permet d'inclure des images au format eps
uniquement.
\item La m�thode "directe" : .tex -> .pdf en utilisant pdflatex. Toutefois, revers de la m�daille, pdflatex ne sait pas reconnaitre les images au format .eps mais vous pouvez
utiliser des jpeg, png, pdf... {\bf C'est la m�thode requise pour compiler ce document.}
\item Remarque : sous geany, il suffit de faire F9 pour g�n�rer le pdf (menu "build" de geany)
\end{itemize}

\section{Quelques commandes}
\subsection{Insertion de figures}

\begin{figure}[!ht]
  \centering
  \includegraphics[height=4cm]{logoLille1}
  \caption{Le logo de l'Universit� Lille 1.}
  \label{fig:ustl}
\end{figure}

Logo de l'USTL (voir fig. \ref{fig:ustl})

Mettre les images dans le r�pertoire \texttt{img}

\subsection{Insertion d'�quations}

\begin{equation}
\int_{t=a}^{t=b}f(t)\;dt = 0 \label{eq:temps}
\end{equation}

D'apr�s l'�quation \ref{eq:temps} ...

\subsection{Insertion de code}

Pour plus d'infos sur le package listings, consulter cette note de bas de page\footnote{
\url{ftp://tug.ctan.org/pub/tex-archive/macros/latex/contrib/listings/listings.pdf}

}

\begin{lstlisting}[frame=trBL]
#include <iostream>

int main() {
    std::cout << "Hello, world!\n";
}
\end{lstlisting}

\subsection{R�f�rences et citations}

\begin{itemize}
\item Pour ajouter une r�f�rence : l'ajouter dans le fichier \verb#biblio.bib# (faire du copier coller avec les exemples qui s'y trouvent)
\item Pour citer une r�f�rence : exemple \verb#\cite{goossens93}# (r�f�rence pr�sente dans \verb#biblio.bib#) pour obtenir \cite{goossens93}
\item {\bf Attention} : pour que les r�f�rences et les citations apparaissent correctement dans le document, il faut ex�cuter \verb#bibtex# avant la compilation latex (sous geany : voir le menu \verb#build#)
\end{itemize}

\section{Doc \LaTeX}
Faire des recherches sur Google ou consulter ce livre tr�s complet
\cite{goossens93}.

\SpecialSection{Conclusion}


\bibliographystyle{myunsrt}
\small
\bibliography{biblio}

\Annex{Annexe 1}

\end{document}
